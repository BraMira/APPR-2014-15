\documentclass[11pt,a4paper]{article}

\usepackage[slovene]{babel}
\usepackage[utf8]{inputenc}
\usepackage{graphicx}
\usepackage{pdfpages}
\usepackage{hyperref}
\usepackage{url}

\pagestyle{plain}

\begin{document}

\title{Poročilo pri predmetu \\
Analiza podatkov s programom R\\}

\author{Mirjam Pergar}
\clearpage\maketitle
\thispagestyle{empty}



\section{Izbira teme}

Naslov mojega projekta je \large\textbf{Splošno zdravstveno stanje in splošno zadovoljstvo z življenjem oseb v Sloveniji.}

Projekt se bo ukvarjal z analiziranjem splošnega zdravstvenega stanja oseb glede na starost in spol in splošnim zadovoljstvom z življenjem oseb v Sloveniji glede na spol, starost, zdravstveno stanje in glede na regije. Pri prvi polovici analiziranja bom predvsem zbrala podatke kako je zdravstveno stanje povezano s starostjo in spolom. Nato bom v drugi polovici preverila kako osebe samoocenijo svoje zadovoljstvo z življenjem. Ali je to povezano s starostjo, spolom, zdravstvenim stanjem in od kod posamezna oseba prihaja? Ali pa med temi faktorji ni nobenih povezav?

Vse podatke bom pridobila s SURS-a iz naslednjih povezav:

\begin{itemize} 
\item \url{http://pxweb.stat.si/pxweb/Dialog/varval.asp?ma=0868510S\&ti=\&path=../Database/Dem\_soc/08\_zivljenjska\_raven/17\_silc\_zdravje/05\_08685\_splosno\_zdravst\_stanje/\&lang=2}

\item  \url{http://pxweb.stat.si/pxweb/Dialog/varval.asp?ma=0868515S\&ti=\&path=../Database/Dem\_soc/08\_zivljenjska\_raven/17\_silc\_zdravje/05\_08685\_splosno\_zdravst\_stanje/\&lang=2}

\item \url{http://pxweb.stat.si/pxweb/Dialog/varval.asp?ma=0872005S\&ti=\&path=../Database/Dem\_soc/08\_zivljenjska\_raven/18\_08720\_silc\_zadovol\_zivljenje/\&lang=2}

\item \url{http://pxweb.stat.si/pxweb/Dialog/varval.asp?ma=0872035S\&ti=\&path=../Database/Dem\_soc/08\_zivljenjska\_raven/18\_08720\_silc\_zadovol\_zivljenje/\&lang=2}

\item  \url{http://pxweb.stat.si/pxweb/Dialog/varval.asp?ma=0872040S\&ti=\&path=../Database/Dem\_soc/08\_zivljenjska\_raven/18\_08720\_silc\_zadovol\_zivljenje/\&lang=2}

Vsi podatki so na voljo v naslednjih oblikah: \textit{.txt, .csv, .htm, .xls}.

V drugi fazi sem dodala še eno tabelo, ki prikazuje podatke za Evropo in sicer pričakovano življenjsko dobo glede na samoocenitev zdravja. 
\item
\url{ http://appsso.eurostat.ec.europa.eu/nui/show.do?dataset=hlth_silc_17&lang=en}

Vsi podatki so na voljo v naslednjih oblikah: .txt, .csv, .htm, .xls.

Za 4. fazo sem uvozila še štiri tabele iz katerih želim po regijah gledati povezave tudi z gospodarsko-ekonomskega vidika.

\item
\url{http://pxweb.stat.si/pxweb/Dialog/varval.asp?ma=0772615S&ti=&path=../Database/Dem_soc/07_trg_dela/10_place/02_07726_kaz_place/&lang=2}

\item
\url{http://pxweb.stat.si/pxweb/Dialog/varval.asp?ma=0772610S&ti=&path=../Database/Dem_soc/07_trg_dela/10_place/02_07726_kaz_place/&lang=2}

\item
\url{http://pxweb.stat.si/pxweb/Dialog/varval.asp?ma=1418806S&ti=Podjetja+po+kohezijskih+in+statisti%E8nih+regijah%2C+Slovenija%2C+letno&path=../Database/Ekonomsko/14_poslovni_subjekti/01_14188_podjetja/&lang=2}

\item
\url{http://pxweb.stat.si/pxweb/Dialog/varval.asp?ma=05C2002S&ti=&path=../Database/Dem_soc/05_prebivalstvo/10_stevilo_preb/10_05C20_prebivalstvo_stat_regije/&lang=2}
\end{itemize}
\subsection{Cilj}
Skozi analize teh podatkov bi rada pokazala:
\begin{itemize}
\item Kakšno je zdravstevno stanje v Sloveniji glede na spol in starost
\item Kako Slovenci samoocenjujemo zadovoljstvo z življenjem glede na zdravstveno stanjel in starost 
\item Podroben prikaz zadovoljstva po regijah
\item Kako se je spremenila pričakovana življenjska doba v Evropi
Na podatke in življenja ljudi je imela velik vpliv gospodarska kriza, zato sem analizirala tudi:
\item Kako so nihale povprečne neto mesečne plače v regijah in občinah
\item Kako se je spremenilo število podjetij,število  zaposlenih v podjetjih in število zaposlenih na podjetje v regiji
\item Kako povezavo imajo te spremenjljivke z zadovoljstvom
\item Kaj lahko napovemo za prihodnost
\end{itemize}
%\newpage
\section{Obdelava, uvoz in čiščenje podatkov}
\subsection{Opis dela}
V drugi fazi projekta sem uvozila v R 5 tabel in 6 grafov. Dve od teh tabel sta bile oblike \textit{.csv} in tri \textit{.htm}, grafi pa so zaradi mojih podatkov vsi stolpični. Tako lahko sedaj v tabelah pogledamo podatke o Splošnem zdravstvenem stanju oseb po spolu in starosti, torej kakšno je zdravje moških in žensk različnih starosti, in Splošnim zadovoljstvom oseb glede na starost,zdravstveno stanje in regije, od koder lahko razbiramo podatke o tem kako so ljudje različnih starosti, različnih zdravstvenih stanj in živeči v različnih regijah države, samoocenili svoje zadovoljstvo z življenjem. V prvem grafu sem predstavila zdravstveno stanje oseb po vseh starostih v letih 2005-2013, v naslednjem grafu pa sem izolirala samo leto 2013, kjer so prikazane vse starostne skupine.Tretji graf prikazuje zdravstveno stanje oseb obeh spolov skupaj v letih 2005-2013, v sledečem grafu sem spet izolirala leto 2013 in pogledala stanja za moške in ženske posebej. V petem grafu je predstavljeno zadovoljstvo z življenjem oseb glede na zdravstveno stanje v letu 2013, kjer sem pogledala zadovoljstvo za moške in ženske posebej, izbrala sem pa tudi mejna zdravstvena stanja; torej zelo dobro in zelo slabo. Zadnji graf pa prikazuje zadovoljstvo z življenjem glede na starost v letu 2013, kjer sem spet pogledala zadovoljstvo posebej za moške in ženske, izbrala pa sem si 3 različne starostne skupine: osebe od 16 do 25 let, 36 do 45 let in 55 do 65 let.

Uvoz sem začela s tabelami \textit{.csv}, ki so bile najlažje za uvoziti, zato sem za njihov uvoz izbrala najbolj zapletene podatke, katere sem pozneje počistila tako, da so namesto dveh glavnih stolpcev kreirala enega, ki vsebuje podatke obeh. Zatem so prišle na vrsto HTML tabele, s katerimi sem se morala veliko bolj potruditi, saj na začetku nisem obvladala vseh ukazov. Po čiščenju nepotrebnih podatkov in preimenovanjih stolpcev in vrstic, sem se lotila grafov. Kar je bilo dokaj enostavno, ko sem vedela, kaj vsaka funckija počne in kako jo uporabiš.

 Pri uvozu sem se srečevala z različnimi težavami, katere sem sproti odpravljala, npr.: kako poteka uvoz iz html-ja,kako oblikovati podatke čim bolj učinkovito in pregledno, kako spremeniš imena vrstic, kako poiskati določene podatke v grafu in jih vključiti v graf ter kodiranje.
\subsection{Opis uvoženih podatkov}
Spodaj bom opisala podatke uvoženih tabel.
\begin{enumerate}
\item \verb+ZdrStarost+ oz. Zdravstevno stanje glede na starost
\begin{enumerate}
\item Imena vrstic: \textit{Zelo dobro, Dobro, Srednje, Slabo, Zelo slabo}
\item \textit{Skupaj (leta 2005-2013)}
\item \textit{Od 16 do 25 let (leta 2005-2013)} 
\item \textit{Od 26 do 35 let (leta 2005-2013)} 
\item \textit{Od 36 do 45 let (leta 2005-2013)}
\item \textit{Od 46 do 55 let (leta 2005-2013)} 
\item \textit{Od 56 do 65 let (leta 2005-2013)} 
\item \textit{66 let in več (leta 2005-2013)} 
\end{enumerate}
\item \verb+ZdrSpol+ oz. Zdravstevno stanje glede na spol
\begin{enumerate}
\item Imena vrstic: \textit{Zelo dobro, Dobro, Srednje, Slabo, Zelo slabo}
\item \textit{Skupaj (leta 2005-2013)} 
\item \textit{Moški (leta 2005-2013)} 
\item \textit{Ženske (leta 2005-2013)} 
\end{enumerate}
\item \verb+ZadStarosti+ oz. Zadovoljstvo z življenjem glede na starost
\begin{enumerate}
\item Imena vrstic: \textit{Moški/Ženske - Starostne skupine - Skupaj, Moški/Ženske - 16-25 let, Moški/Ženske - 26-35 let, Moški/Žen\-ske - 36-45 let, Moški/Ženske - 46-55 let, Moški/Ženske - 56-65 let, Moški/Ženske - 66 ali več let}
\item \textit{Povsem nezadovoljen (leta 2012,2013)} : urejenostna spremenljivka
\item \textit{Nezadovoljen (leta 2012,2013)} : urejenostna spremenljivka
\item \textit{Zadovoljen (leta 2012,2013)} : urejenostna spremenljivka
\item \textit{Zelo zadovoljen (leta 2012,2013)} : urejenostna spremenljivka
\item \textit{Neznano (leta 2012,2013)} 
\item \textit{Povprečje (leta 2012,2013)} 
\end{enumerate}
\item \verb+ZadStanje+ oz. Zadovoljstvo z življenjem glede na zdravstveno stanje
\begin{enumerate}
\item Imena vrstic: \textit{Moški/ženske - Zelo dobro, Moški/ženske - Dobro, Moški/ženske - Srednje, Moški/ženske - Slabo, Moški/žen\-ske - Zelo slabo}
\item \textit{Povsem nezadovoljen (leta 2012,2013)} : urejenostna spremenljivka
\item \textit{Nezadovoljen (leta 2012,2013)} : urejenostna spremenljivka
\item \textit{Zadovoljen (leta 2012,2013)} : urejenostna spremenljivka
\item \textit{Zelo zadovoljen (leta 2012,2013)} : urejenostna spremenljivka
\item \textit{Neznano (leta 2012,2013)} 
\item \textit{Povprečje (leta 2012,2013)} 
\end{enumerate}
\item \verb+ZadRegije+ oz. Zadovoljstvo z življenjem glede na statistične regije
\begin{enumerate}
\item Imena vrstic: \textit{Slovenija, Pomurska,Podravska, Koroška, Savinjska, Zasavska, Spodnjeposavska, Jugovzhodna Slovenija, Osrednjeslovenska, Gorenjska, Notranjsko-kraška,Goriška, O\-bal\-no-kraška}
\item \textit{Povsem nezadovoljen (leta 2012,2013)} : urejenostna spremenljivka
\item \textit{Nezadovoljen (leta 2012,2013)} : urejenostna spremenljivka
\item \textit{Zadovoljen (leta 2012,2013)} : urejenostna spremenljivka
\item \textit{Zelo zadovoljen (leta 2012,2013)} : urejenostna spremenljivka
\item \textit{Neznano (leta 2012,2013)} 
\item \textit{Povprečje (leta 2012,2013)} 
\end{enumerate}
\end{enumerate}

\subsection{Prestavitev in interpretacija grafov}
\includepdf[pages={1,2}, nup=1x2]{../slike/starosti.pdf}

\subsubsection{Interpretacija}
\verb+Graf 1+ prikazuje zdravstveno stanje oseb po vseh starostih v letih 2005 do 2013. Opazimo, da zdravstveno stanje skozi leta ni le padalo ali rastlo, ampak je nihalo v obe smeri. Oseb z \textit{Zelo dobrim} zdravstvenim stanjem je v vedno več, saj jih je leta 2013 bilo že preko 20. Število oseb  z \textit{dobrim} zdravstvenim stanjem(največ ljudi je v tej skupini) je v letih nihalo, vendar jih je leta 2013 bilo več kot leta 2005. Število oseb s \textit{srednjim} zdravstvenim stanjem je padlo, medtem, ko je s textit{slabim} in \textit{zelo slabim} stagniralo, vendar je število v letu 2013 manjše kot v 2005, kar pomeni, da je oseb s šibkim zdravjem vedno manj.
\verb+Graf 2+ prikazuje zdravstvena stanja po starostnih skupinah v letu 2013. Iz grafa lahko razberemo nekatere informacije, ki potrjujejo domneve, da so mladi(\textit{16-25 let}) zelo zdravi, najstarejši(\textit{66 in več let}) ljudje pa najmanj zdravi. V drugi starostni skupini \textit{26-35 let} se število oseb s dobrim in zelo slabim zdravstvenim stanjem poveča, medtem ko se z zelo dobrim zmanjša. Ta razlika se poveča še bolj v skupini \textit{36-45 let}, v \textit{46-55 let} pa je oseb s srednjim zdravstvenim stanjem že toliko kot z dobrim.
\includepdf[pages={1,2}, nup=1x2]{../slike/spoli.pdf}
\subsubsection{Interpretacija}
V \verb+Grafu 4+ lahko vidimo, kako se je spreminjalo zdravstveno stanje obeh spolov skupaj v letih 2005 do 2013. Število oseb z  \textit{zelo dobrim} stanje se je povečalo, medtem ko se je število \textit{srednjih, slabih, zelo slabih} zmanjšalo. Število \textit{dobrih} pa je skozi leta nihalo, vendar se je nekoliko povečalo v primerjavi z letom 2005.
\verb+Graf 4+ prikazuje zdravstvena stanja oseb ločena po spolu v letu 2013, tako lahko primerjamo moške in ženske. Vidimo, da pri moških prevladuje \textit{dobro} zdravstveno stanje, sledi \textit{zelo dobro} in \textit{srednje}, medtem ko pa pri ženskah \textit{dobremu} sledi \textit{srednje} in nato \textit{zelo dobro}. Pri ženskah je tudi opaziti, da je več \textit{slabih} oseb kot pri moških. Iz tega bi lahko interpretirali, da so ženske manj zdrave kot moški.



\makebox[\textwidth][c]{
 \includegraphics[scale=0.8]{../slike/zad_stanje.pdf}
}
\subsubsection{Interpretacija}
\verb+Graf 5+ prikazuje zadovoljstvo z življenjem glede na zdravstveno stanje ljudi v letu 2013. Za prikaz sem si izbrala mejni skupini pri moških in ženskah, torej \textit{zelo dobro} in \textit{zelo slabo}.  Iz grafa lahko vidimo, da so najbolj zadovoljni moški in ženske z \textit{zelo dobrim} zravstvenim stanjem, medtem, ko so najmanj zadovoljni moški z \textit{zelo slabim} zdravstvenim stanjem, ki jim sledijo ženske z \textit{zelo slabim} zdravstvenim stanjem.
\makebox[\textwidth][c]{
\includegraphics[scale=0.8]{../slike/zad_star.pdf}
}
\subsubsection{Interpretacija}
\verb+Graf 6+ prikazuje zadovoljstvo z življenjem glede na starost v letu 2013. Za skupine za prikaz sem sem izbrala moške in ženske starostnih skupin \textit{16-25 let, 36-45 let} in \textit{55-65 let}. Iz grafa preberemo, da so najbolj zadovoljne ženske med 16-25 letom, sledijo jim moški med 16-25 leti. Najmanj zadovoljni pa so pričakovano moški in ženske med 55-65 let. Pri vseh starostnih skupinah je največ jih izbralo zadovoljstvo med 7 in 8 procenti. Zaključimo lahko, da so mladi(16-25 let) najzadovoljnejši, najstarejši (56-65 let) pa najmanj zadovoljni.
\section{Analiza in vizualizacija podatkov}
\subsection{Opis dela}
V tretji fazi projekta sem v R uvozila 7 zemljevidov in eno novo tabelo. Pet od teh zemljevidov so zemljevidi Slovenije, ostala dva pa zemljevida Evrope. Mojo analizo sem začela tako, da sem na internetu poiskala zemljevid in se spoznala z že pripravljenimi funkcijami v R-u. Na prvem zemljevidu sem predstavila povprečno zadovoljstvo z življenjem v letu 2013 po regijah iz katerega je razvidno kako so ljudje v povprečju zadovoljni z življenjem.

Na naslednjih zemljevidih sem predstavila podatke malo bolj podrobno, saj na štirih zemljevidih prikazujem ljudi, ki so povsem nezadovoljni z življenjem, nezadovoljni, zadovoljni in zelo zadovoljni. Tako je barvno lepo vidno, kje je zadovoljstvo boljše in kje slabše.

Ker sem po uvozu teh zemljevidov bila mnenja, da bi bilo potrebno predstaviti še kak podatek več tudi na evropski ravni, sem se odločila da uvozim novo tabelo. Nova tabela je bila oblike \textit{.csv}, dobila pa sem jo s strani \url{http://appsso.eurostat.ec.europa.eu/nui/show.do?dataset=hlth_silc_17&lang=en}. Prikazuje podatke za pri\-ča\-ko\-va\-no življenjsko starost po vseh državah Evrope. To sicer ni enako kot zadovoljstvo z življenjem, vendar je pričakovana življenjska doba tudi dober pokazatelj zdravja in zadovoljstva z življenjem. Po uvozu tabele za moške in ženske za leto 2004 in 2012, sem izračunala povprečja za ti dve leti, za lažji prikaz na zemljevidu.  Tako lahko na dveh zemljevidih vidimo ta dva podatka, seveda je za leto 2012 več podatkov kot za leto 2004 in lahko iz njega dobimo več zaključkov.

V 3. fazi nisem imela večjih težav, malo več časa sem porabila le za končne podrobnosti pri zemljevidu Evrope.
\subsection{Opis novih uvoženih podatkov}
Za prestavitev podatkov tudi na evropski ravni sem uvozila tabelo Povprečnih pričakovanih življenjskih dob v letu 2004 in 2012. Na začetku se je tabela imenovala \verb+Evropa+ in je imela spremenljivke \textit{time, geo, sex} in \textit{value}. Po preurejanjih sem oblikovala novo tabelo.
\begin{enumerate}
\item \verb+ZivZad+ oz. Povprečna pričakovana življenjska doba v Evropi
\begin{enumerate}
\item Imena vrstic: \textit{31 evropskih držav}
\item \textit{Moški/Ženske 2004/2012} 
\item \textit{Povprečje 2004/2012} 
\end{enumerate}
\end{enumerate}

\subsection{Predstavitev in interpretacija zemljevidov}
\makebox[\textwidth][c]{
\includegraphics{../slike/zemljevid1.pdf}
}
\subsubsection{Interpretacija}
Na \verb+Zemljevidu 1+ prikazuje podatke o povprečnem zadovoljstvu z življenjem v letu 2013 po regijah Slovenije. Zemljevid je pobarvan z različnimi odtenki rumene barve, kjer močnejši odtenek pomeni večje zadovoljstvo, šibkejši odtenek pa manjše zadovoljstvo. Tako lahko najhitreje ugotovimo, kje je kakšno zadovoljstvo, npr. v Osrednjisloveniji in na Gorenjskem je odtenek najmočnejši, medtem ko je na Zasavskem in v Pomurju najšibkejši. Če želimo vedeti točne številke, jih enostavno pogledamo pod imeni regij in vidimo, da je zadovoljstvo res največje v Osrednjisloveniji in najslabše na Zasavskem. Povsod drugje se giblje med 7 in 6.9 (\%).
\includepdf[pages={1-4}, nup=2x2]{../slike/zemljevid2.pdf}
\subsubsection{Interpretacija}
Na teh štirih zemljevidih lahko vidimo, koliko ljudi v posemznih regijah je \textit{povsem nezadovoljnih, nezadovoljnih, zadovoljnih} in \textit{zelo zadovoljnih} z življenjem, pomagamo si z barvno lestvico na desni strani. Opazimo, da je največ povsem nezadovoljnih ljudi na Zasavskem, sledi Obalno-kraška, najmanj pa v Spodnjeposavski, Osrednjeslovenski in Goriški. Največ ljudi nezadovoljnih z življenjem je spet na Zasavskem, sledi Pomurje, najmanj pa v Osrednjisloveniji. Največ zadovoljnih ljudi je v Osrednjisloveniji in v SPodnjeposavski, najmanj pa na Zasavskem. In nazadnje največ zelo zadovoljnih ljudi je v Osrednjisloveniji, Gorenjski in Podravski regiji, najmanj pa v Pomurju. Iz teh zemljevidov pridemo do enakih rezultatov kot jih prikazuje \verb+Zemljevid 1+, vendar pa lahko opazimo še to, da je največ ljudi izbralo možnost \textit{zadovoljni}, sledi \textit{nezadovoljni, zelo zadovoljni} in \textit{povsem nezadovoljni} iz česa lahko sklepamo, da je zadovoljstvo v Sloveniji prevladujoče nad nezadovoljstvom.
\includepdf[pages={1-2},nup=1x2]{../slike/zemljevidE.pdf}
\subsubsection{Interpretacija}
Za \verb+Zemljevid 6 in 7+ sem uvozila nove podatke o evropskih državah, natančneje o povprečni pričakovani starosti v letu 2004 in 2012. Želela sem, da bi zemljevida predstavila spremembov pričakovani starosti, zato sem izbrala najmanjše in največje možno leto. Vendar, kar je v letu 2004 podatkov za veliko držav manjkalo, je za nekatere držav e težko opisati spremembe, lahko jih le primerjamo s podatki držav, ki so bili na voljo. Tako za leto 2004 opazimo, da je bila najvišja pričakovana starost(72-76 let) v skandinavskih državah in razvitih državah srednje Evrope kot tudi v Italiji in Grčiji, nanižja pa v Estoniji, na Portugalskem in v Španiji, kjer je bila nižja od 70-ih let. Na \verb+Zemljevidu 7+ za leto 2012, lahko prvo že opazimo na barvni lestvici, da se je najvišja povprečna pričakovana starost dvignila z 76 na 80 let. V skandinavskih državah sprememba ni velika, v državah srednje Evrope in Italije in Grčije se starost skorajdi ni spremenila. Nasportno pa se je dvignila v Estoniji, na Portugalskem in v Španiji, kjer je že okoli 75 let. V državah vzhodne Evrope, vidimo, da je starost primerljiva s starostjo v Estoniji, zato bi lahko sklepali, da se je v teh državah starost enako spremenila. Če pogledamo Slovenijo, vidimo, da je povprečna pričakovana starost boljša kot v državah vzhodne Evrope, vendar slabša kot v državah srednje in severne Evrope. Pri nas je povprečna pričakovana starost okoli 71 let. Iz teh dveh zemljevidov lahko torej sklepamo, da je povprečna pričakovana starost boljša v bolj razvitih državah, vendar pa to ni pravilo.
%\includegraphics{../slike/povprecna_druzina.pdf}

\section{Napredna analiza podatkov}
\subsection{Opis postopka dela}
Za 4. fazo projekta sem se po nasvetu profesorja osredotočila na analizo podatkov za regije. Ker sem za regije imela le eno tabelo, sem uvozila še štiri tabele s katerimi sem si pomagala pri analizi. Ko sem iskala ideje za analizo podatkov, sem se začela osredotočati na gospodarske in ekonomske vidike, zato se tudi nove tabele osredotočajo na to tematiko.

Pri delu nisem imela večjih težav, saj sem vse težave sproti odpravljala.
\subsection{Prestavitev novo uvoženih podatkov}
Prvo tabelo, ki jo bom opisala, sem uvozila samo za pomoč izračuna normalnih vrednosti, ostale pa sem uporabili pri analizi in napovedi podatkov. Pri vseh tabelah(razen druge) so imena vrstic: \textit{Slovenija, Pomurska,Podravska, Koroška, Savinjska, Zasavska, Spodnjeposavska, Jugovzhodna Slovenija, Osrednjeslovenska, Gorenjska, Notranjsko-kraška, Goriška, Obalno-kraška}
\begin{enumerate}
\item \verb+Prebivalstvo+ oz. Število prebivalcev v regijiah
\begin{enumerate}
\item \textit{2005-2014}
\end{enumerate}
\item \verb+PovpPl+ oz. Povprečne mesečne neto plače v vseh slovenskih občinah
\begin{enumerate}
\item Imena vrstic: \textit{211 slovenskih občin}
\item \textit{2005-2013} 
\item \textit{Povprečje} 
\end{enumerate}
\item \verb+PovpR+ oz. Povprečne mesečne neto plače v regijah
\begin{enumerate}
\item \textit{2005-2013} 
\end{enumerate}
\item \verb+Podjetja+ oz. Podjetja po statistničnih regijah
\begin{enumerate}
\item \textit{Število podjetij 2008-2013} : imenska spremenljivka
\item \textit{Število oseb, ki delajo 2008-2013} 
\item \textit{Prihodek 2008-2013} 
\item \textit{Število oseb, ki delajo na podjetje v regiji 2008-2013} 
\end{enumerate}
\end{enumerate}
\subsection{Predstavitev in interpretacija grafov}
Za 4. fazo sem narisala grafe, ki predstavljajo bolj gospodarske in ekonomske vidike, saj bi rada zadovoljstvo z življenjem povezala z tem vidikom. Tako sem prikazala grafe s podatki o plačah in podatkih o podjetjih v regijah.
\includepdf[pages={1,2}, nup = 1x2]{../slike/Povprecne_place.pdf}
\subsubsection{Interpretacija}
Na \verb+Grafu 7 in 8+ prikazujem povprečne mesečne neto plače v regijah v letih 2008 do 2013. Želela sem pokazati, kako so se v letih spreminjale in ugotoviti v katerih letih so bile največje spremembe. Na \verb+Grafu 7+ vidimo spremnjanje plač v regijah in katere regije imajo kakšne plače. Vidimo lahko, da ima Osrednjeslovenska regija občutno višje plače kot ostale regije. Nanižje plače pa imata Pomurska in Notranjsko-kraška regija. Opazimo lahko, da so plače na začetku hitro rastle, nato so v letih 2009 in 2011 začele stagnirat in rast zelo počasi, kar je seveda posledica gospodarske krize, ki je vplivala na močan padec gospodarske aktivnosti in na gibanje na trgu dela. 
Na \verb+Grafu 8+ pa sem pogledala, kakšne so plače v večjih mestih regij. Za boljšo predstavo, sem poiskala mesti, kjer je povprečje plač največje in najmanjše, ter tudi njiju predstavila na grafu kot nekakšne "meje". Opazimo lahko, da so spremembe enake, v letu 2013 pa plače skorajde več ne rastejo, kar ni preveč "rožnata" analiza.
\includepdf[pages={1,2,3},nup=1x3]{../slike/Podjetja.pdf}
\subsubsection{Interpretacija}
V \verb+Grafih 9-11+ sem predstavila podatke o podjetjih. Zanimalo me je predvsem kako se je spreminjalo število podjetij, število zaposlenih v podjetjih in število oseb, ki delajo na podjetje v regiji. Čeprav, me ni zanimalo, v kateri regiji je več podjetij in več zaposlenih, je na grafih možno videti tudi to. Iz \verb+Grafa9+ lahko vidimo, da je število podjetij v letih 2008-2013 rastlo, kar je nekoli presenetljiva informacija, glede na to, da je gospodarska aktivnost zaradi krize padla. \verb+Grafa 10 in 11+ sta boljši prikaz vpliva gospodarske krize, saj je zaposlenost padla in prav tako število oseb zaposlenih na podjetje v regiji, kar pomeni, da je veliko oseb ostalo brez služb. 
\subsection{Predstavitev in interpretacija napovedi}
Za 4. fazo sem želela prikazati napovedi zadovoljstva z življenjem, kjer sem na grafih najprej predstavila, kakšna prileganja so možna in nato s temi prileganje napovedala kako se spreminja zadovoljstvo glede na spremenljivke. Uporabila sem regresijska prileganja z modeli: linearni, kvadratični, loess(za lokalno prileganje) in lowess(prileganje z zlepki)
\includepdf[pages={1,2},nup=1x2]{../slike/Napovedi1.pdf}
\subsubsection{Interpretacija}
\par \verb+Graf 12+ prikazuje povezavo med zadovoljstvom z življenjem (x os)  in številom podjetij(y os) v regijah v letih 2012 in 2013. Leto 2012 predstavljajo polni črni krogi, 2013 pa črni krogi. Za povezave sem pogledala linearno, kvadratično in loess prileganje.
\par Linearno prileganje je predstavljeno z modro barvo in enačba premice je $y= -0.26665864 + 0.04968638 * zad$, kjer je zad spremenljivka vrednosti zadovoljstva z življenjem. Iz linearnega modela lahko sklepamo, da je višje število podjetij je zadosten pogoj za višje zadovoljstvo, vendar pa so izjeme, kjer je število podjetij visoko, zadovoljstvo pa ne. Za napoved, brez izjem, bi torej lahko rekli, da je zadovoljstvo višje, če je število podjetij višje. To se seveda sliši razumljivo, saj naj bi število podjetij namigovalo na to, da je tudi več ljudi zaposlenih oz. manj brezposelnih, kar sigurno poveča zadovoljstvo, vendar kot smo videli na grafih in zemljevidih, v določenih regijah so plače manjše, kar lahko vpliva na zadovoljstvo. Kako natančen je linearen model lahko preverimo s vsoto residualov: $0.004113540$.

Kvadratični model je predstavljen z vijolično črto in njega formula je $y= -3.26823991 - 0.06165265 * zad2 + 0.91031486 *zad^2$. Tudi kvadratični model prikazuje isto razmerje kot linearen, vendar pa napoveduje drugačno situacijo. Pri njem je razvidno, da je rast števila podjetij omejena navzgor, kjer je zadovoljstvo med navečjim in najslabšim. Napoveduje pa, da je pri enakem številu podjetij lahko zadovoljsto boljše ali slabše, kar je vidno v \verb+Grafu 12+, kjer je ponekod za isto število podjetij velika razlika v zadovoljstvu. Natančnost kvadratičnega modela je: $0.003980568$, kar je boljše kot linearen model.

Prikazala sem tudi loess model. Model za lokalno prilagajanje, se uporablja predvsem za glajenje krivulj. Podano imamo množico točk x in želimo povleči zglajeno krivuljo, ki se bo najbolje prilagajala našim točkam. Na našem grafu lahko vidimo, da je oranžna krivulja, taka krivulja, ki se našim točkam najbolje lokalno prilega. Graf se obnaša skoraj povsod podobno kot linearen model, razen na desnem delu, kjer je pri določeni vrednosti zadovoljstva, večje število podjetij. Vendar pa s tem modelom ne moremo napovedovati podatkov. Lahko pa preverimo kako natančen je: $0.003055163$, kar pomeni, da je nabolj natačen izmed teh treh modelov.

\includepdf[pages={1,2},nup=1x2]{../slike/Napovedi2.pdf}
\subsubsection{Interpretacija}
\par \verb+Graf 13+ prikazuje povezavo med zadovoljstvom z življenjem (x os) in številom zaposlenih (y os) v regijah v letih 2012 in 2013. Leto 2012 predstavljajo polni črni krogi, 2013 pa črni krogi. Za povezave sem pogledala linearno, kvadratično in loess prileganje.
\par Linearno prileganje je predstavljeno z modro črto in enačba premice je $y= -1.4462077 + 0.2556197 *zad$, kjer je zad spremenljivka vrednosti zadovoljstva z življenjem. Linearen model pove, da je zadovoljstvo višje, če je število zaposlenih višje, kar bi tudi predpostavili, vendar na grafu vidimo, da najvišje zadovoljstvo nimajo krogci z največjim številom zaposlenih, tako da je število zaposlenih definitivno pogoj za višje zadovoljstvo, vendar pa ni zagotovilo. Za prihodnost bi torej lahko napovedali, da bo zadovoljstvo višje, če bo število zaposlenih večje. Natančnost linearnega modela je: $0.12005572$.
\par Kvadratično prileganje je na grafu predstavljeno z vijolično barvo in njegova formula je $y= -6.26722213 -0.09902391*zad2 + 1.63792525 *zad^2$. Na grafu je kvadratični model skoraj identičen linearnemu, razlikuje se le v napovedi, kjer napove podobno kot v \verb+Napoved 1+. Torej je pri največji zaposlenosti, zadovoljsto ni maksimalno, prav tako pri dveh istih vrednostih zaposlitve, je lahko zadovoljstvo različno, boljše ali slabše. Natančnost modela je: $0.11971269$, kar pomeni, da je boljši kot linearen model.
\par Z oranžno krivuljo sem prikazala še model za lokalno prilagajanje - loess, ki se našim točkam najbolje lokalno prilega. Vidimo, da je na začetku malo bolj položen in ima en skok. Natančnost modela je: $0.05474617$, kar pomeni, da je najbolj natančen izmed teh treh modelov.
\includepdf[pages={1,2},nup=1x2]{../slike/Napovedi3.pdf}
\subsubsection{Interpretacija}
\par \verb+Graf 14+ prikazuje povezavo med zadovoljstvom z življenjem (x os) in povprečno mesečno neto plačo (y os) v regijah v letih 2012,2013. Leto 2012 predstavljajo polni črni krogi, 2013 pa črni krogi. Za povezave sem pogledala linearno, kvadratično, loess prileganje in prileganje z zlepki.
\par Linearno prileganje je predstavljeno z modro črto in njegova enačba je $y=114.0710 + 120.2186*zad$, kjer je zad spremenljivka vrednosti zadovoljstva z življenjem. Na tem grafu je linearna povezava sicer bolj šibka, vendar je razvidno, da je višja plača nek faktor v višjem zadovoljstvu, vendar pa ni zagotovilo za višje zadovoljstvo, kar se vidi pri nekaterih krogcih v grafu. Napoved za prihodnost v linearnem modelu je torej, da bo ob višji plači višje zadovoljstvo. Natančnost modela je: $52355.07$.
\par  Kvadratično prileganje je na grafu predstavljeno z vijolično črto in njegova enačba je $y=-2255.64756 -48.67415*zad + 799.67624*zad^2$. Graf je podoben linearnemu in prikazuje isto situacijo, vendar pa napoveduje drugačno sliko, podobno tistima iz prejšnjih napovedi, kjer imata dva krogca isto plačo, ampak različni zadovoljstvi, boljšo in slabšo. Natančnost modela je:$52272.19$, kar je malo natančneje od linearnega.
\par Z oranžno krivuljo sem prikazala še model za lokalno prilagajanje - loess, ki se našim točkam najbolje lokalno prilega. Vidimo, da ima njegov graf več skokov in ni preveč linearen. Natančnost modela je: $27649.77$, kar pomeni, da je najbolj natančen izmed teh treh modelov.
%\par Pri tem grafu sem si pogledala kako deluje še najnatančnejša metoda - prileganje z zlepki(lowess), prikazana z zeleno %barvo. Funkcija lowess nam vrne točke, ki nam predstavljajo "zglajen" graf.
\subsection{Zaključek}
Skozi analizo moje teme, sem prišla do marsikaterega zaključka in odgovorila na vsa vprašanja, ki sem si jih zadala. Izvleček teh zaključkov je:
\begin{itemize}
\item Zdravstveno stanje glede na starost v večini dobro. Najboljše zdravstevno stanje, v letu 2013,  imajo 16-25 letniki, najslabše pa 66+ letniki. Zdravstveno stanje glede na spol je tudi večinoma dobro, v letu 2013, je bilo več moških z zelo dobrim zdravstvenim stanjem kot žensk.
\item Najbolj zadovoljni z življenjem glede na zdravstveno stanje so seveda moški in ženske z zelo dobrim zdravstveno stanjem, najmanj zadovoljni so pa moški z zelo slabim stanjem, katerim sledijo ženske z zelo slabim stanjem. Glede na starost pa ni presenetljivo, da so najbolj zadovoljne ženske med 16-25 leti in moški med 16-25 leti. Najmanj zadovoljni pa moški med 55-65 leti.
\item Najbolj zadovoljni so ljudje v Osrednjeslovenski regiji (povprečje 7.2), sledi Gorenjska (povprečje 7.1). Najmanj zadovoljni pa v Zasavski regiji (povprečje 6.6), sledi Pomurska regija (povprečje 6.7). V vseh ostalih regijah je zadovoljstvo 7 ali 6.9. Največ ljudi je zadovoljnih z življenjem, sledi nezadovoljni, zelo zadovoljni in nezadovoljni.
\item Povprečna pričakovana življenska doba v Evropi se je v letih 2004-2012 povišala v večini držav, najvišja je še vedno v skandinavskih državah, nanižja v v vzhodni Evropi.
\item Povprečna mesečna neto plača je v regijah naraščala, vendar je imela v večini regij padec v letu 2009 in 2011, zaradi gospodarske krize, katere posledica je zelo počasna rast. Tudi v občinah je podobna situacija.
\item Število podjetij se je v letih 2008-2013 povečevalo, medtem ko se je število zaposlenih nižalo. Zelo je padlo tudi število oseb, ki delajo na podjetje v regiji. 
\item Povezava med zadovoljstvom in številom podjetij: Večanja števila podjetij ima vpliv na večje zadovoljstvo, vendar ni zagotovilo. Povezava med zadovoljstvom in številom zaposlenih: Večanje zaposlitve in vpliv na višje zadosovljstvo, vendar so izjeme, ko je višje zadovoljstno nedovisno od števila zaposlenih. Povezava med zadovoljstvom in povprečno mesečno neto plačo: Obstaja šibka povezava, višja plača je definitivno pogoj za višje zadovoljstvo, vendar ni zagotovilo.
\item Za prihodnost zelo natančnih napovedi ne moremo narediti, lahko samo napovemo, da bi se teoretično ob ob večjem številu podjetij, zaposlenih in višjih plačah zadovoljstvo povečalo.
\end{itemize}
\end{document}
